	\pagenumbering{roman}

		% \addcontentsline{toc}{section}{Certificate of Approval}
		% \large
		% 	\begingroup
		% 		\let\clearpage\relax
		% 		\chapter*{Certificate of Approval}
		% 	\endgroup


		% \normalsize
		% 	Your text here...
		% \break



		% \addcontentsline{toc}{section}{Copyright}
		% \large
		% 	\chapter*{Copyright}
		% \normalsize
		% 	Your text here...
		% \break

	
	    \addcontentsline{toc}{section}{Acknowledgement}
		 \large
            \chapter*{Acknowledgement}
            % \addcontentsline{toc}{chapter}{Acknowledgement}
            \normalsize
            
            We would like to express our heartfelt gratitude to our HOD, \textbf{Er. Dinesh Man Gothe}, for providing us with the opportunity and the confidence to act on our will to work on this project and proposal. Also, not forgetting the great help of our supervisor, \textbf{Er. Mukesh Kumar Pokharel}, for providing us with the vision to work on our project.


            \vspace{1 cm}


            \begin{tabular}{ll}
                \centering
                \textbf{Chirag Khatiwada} & KCE077BCT001 \\
                \textbf{Bishesh Pokharel} & KCE077BCT014 \\
                \textbf{Mahim Rawal} & KCE077BCT019 \\
                \textbf{Rowel Maharjan} & KCE077BCT027 \\
            \end{tabular}

                   

   
		\break
		
		\addcontentsline{toc}{section}{Abstract}
		\large
			\chapter*{Abstract}
		\normalsize
			Decoding text from a speaker's facial movements, or lip reading, is a skill that has great potential for people with speech or hearing impairments. In addition to improving communication, it makes captioning easier in difficult audio situations, such as crowded spaces or far-off speakers. Automated lip reading systems have been made possible by recent advances in computer vision and machine learning. In order to improve the accuracy of visual voice recognition, this research project suggests a hybrid architecture that combines Long Short-Term Memory (LSTM) and Convolutional Neural Networks (CNN), taking advantage of their respective advantages. To ensure robustness and generalization, the hybrid model will be trained on an extensive dataset that spans a variety of speakers, languages, and environmental situations. Model parameters will be optimized using fine-tuning processes to provide adaptability in a variety of lip reading circumstances. In order to simplify the implementation process, the project is limited to visual data.
            \break	\break
            \textbf{Keywords}: Lip reading, Convolutional Neural Networks (CNN),
            Long Short-Term Memory (LSTM) networks

		\break


		\tableofcontents
		
		\addcontentsline{toc}{section}{List of Tables}
		\listoftables
		\break
		\pagebreak

		\addcontentsline{toc}{section}{List of Figures}
		\listoffigures
		\break
	
	
	
		\addcontentsline{toc}{section}{List of Symbols and Abbreviation}
		\Large
			\begingroup
				\let\clearpage\relax
				\chapter*{List of Symbols and Abbreviation}
			\endgroup
		\normalsize

            \begin{tabular}{p{2cm} p{10cm}}
                AL & Automatic Lipreading \\
                ASR & Automatic Speech Recognition \\
                Bi-GRU & Bi-directional Gated Recurrent Unit \\
                CNN & Convolutional Neural Networks \\
                GB & Giga Byte\\
                GCP & Google Cloud Platform\\
                GPU & Graphics Processing Unit\\
                VM & Virtual Machine\\
                SAT & Speaker Adaptive Training \\
                RNN & Recurrent Neural Network \\
            \end{tabular}
		\break
		\pagebreak
		
	