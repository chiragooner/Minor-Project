     \chapter{Introduction}
        \pagenumbering{arabic}
        \section{Background Introduction}People often communicate through hearing and vision, that is, through voice signals and visual signals. Speech signals often contain more information than visual signals, so many studies have focused on Automatic Speech Recognition (ASR). Although automatic speech recognition (ASR) technology is mature, there are still some unsolved problems, such as how to accurately identify what the speaker is saying in a noisy environment. Lipreading is a visual speech recognition technology that recognizes the speech content based on the motion characteristics of the speaker’s lips without speech signals. Therefore, lipreading can detect the speaker’s content in a noisy environment, even without a voice signal.
Machine learning methods have a great impact on social progress in recent years, which promoted the rapid development of artificial intelligence technology and solved many practical problems. Automatic lip-reading technology is one of the important components of human–computer interaction technology and virtual reality (VR) technology. It plays a vital role in human language communication and visual perception. This project investigates the task of speech recognition from video without audio. The input data to our algorithm is sequences of still images taken from frames of video. We use models to output one of 10 words that are spoken by a face in the input images. We explore and combine a number of different models including CNNs, RNNs and existing publicly available pretrained models to assist in mouth recognition. 
        
       
        \section{Motivation}
        The primary objective of developing lip reading software is to increase accessibility by utilizing technology as a tool to remove obstacles to communication for those who have speech or hearing problems. By creating software that can identify and convert lip movements into text, we hope to make it easier for those who are hard of hearing to understand spoken language. In addition, those with speech impairments can now communicate clearly and effectively thanks to this technology. Not only may lip reading increase the accessibility of media for those who have hearing loss, but it can also be used to automatically caption videos, which is useful when the audio quality is not good. The primary goal is to employ technology to enhance the overall standard of living for those with speech or hearing impairments.
        
        \newpage
        \section{Problem Definition}
        At present, the ASR can reach a very high recognition rate without severely damaging the speech signal and also can be used in many practical fields. Visual speech recognition is a technology that recognizes the speech content by lip movement characteristics on no speech signal. The information received by the voice channel is two dimensional. Compared with the one-dimensional voice information received by the voice channel, the visual information often contains more redundant information. So visual speech recognition has always been a difficult problem to solve. Visual speech technology is also known as Automatic Lipreading (ALR), which infers the speech content according to the movement of lips in the process of speaking. In real world, there are people with hearing impairment. They communicate through sign language or observing through people’s lip movements. But gesture language has problems such as being difficult to learn and understand, and inadequate expression skills. Therefore, ALR technology can help people with hearing impairment communicate with others better to some extent. Also in noisy environments, the speech signal is easily interfered with by the surrounding noise, resulting in the reduction of recognition rate. However, the visual information needed for ALR will not be affected, so ALR can improve the recognition effect of speech recognition in noisy environments. In the field of security, first of all, with the popularity of face recognition technology, there are many attacks against face recognition system, such as photos, video playback, and 3D modeling, etc. Adding lip features can further improve the security and stability of the security system. In the field of vision synthesis, traditional speech synthesis can only synthesize a single voice, and lipreading technology can generate high-resolution speech scene video of specific people. Besides, in sign language recognition, lip movements are also combined to better understand the content of sign language or improve the accuracy of sign language recognition.
        \section{Goals and Objectives}
        \textbf{To act as hearing aid for deaf people and voice for mute people.}      